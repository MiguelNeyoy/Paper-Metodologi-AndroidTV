\documentclass[conference, onecolumn]{IEEEtran} % Define el estilo IEEE de doble columna
\usepackage[utf8]{inputenc}      % Permite escribir tildes y ñ directamente
\usepackage[spanish,es-tabla]{babel} % Configura el idioma español y que diga "Tabla" en vez de "Cuadro"
\usepackage{cite}                % Para las citas bibliográficas [1]
\usepackage{graphicx}            % Para insertar imágenes
\usepackage{amsmath,amssymb}     % Para símbolos matemáticos
\usepackage{url}                 % Para las URLs en las referencias


\begin{document}

\title{Metodología para el desarrollo de transiciones visuales secuenciales en Android TV}

\author{\IEEEauthorblockN{Miguel Ángel Neyoy Ceniceros, Jesús Rodolfo Carvajal Lizárraga,\\
        Juan Daniel Vázquez García, Modesto Alonso Alcaraz Duran}
    \IEEEauthorblockA{\textit{Facultad de Informática Mazatlán} \\
        \textit{Universidad Autónoma de Sinaloa}\\
        Mazatlán, México \\
    }}

\maketitle % Esto genera el bloque de título
% --- RESUMEN (Español) ---
\begin{abstract}
    Esta investigación establece una metodología estructurada para el desarrollo de aplicaciones con
    transiciones visuales secuenciales en el entorno de Android TV. El trabajo responde a la fragmentación
    del mercado de Smart TV y la necesidad de enfoques de desarrollo que aseguren rendimiento y
    compatibilidad. La metodología propuesta emplea Kotlin y el framework Jetpack Compose for TV, los
    cuales son recomendados por Google, y se validan mediante un prototipo de carrusel de imágenes. La
    implementación demostró la importancia de la gestión del estado reactivo (remember, mutableStateOf)
    para mantener la estabilidad de las transiciones visuales. Se confirmó la efectividad de la carga asíncrona
    de recursos remotos (librería Coil) para garantizar un flujo visual continuo, evitando bloqueos del hilo
    principal. Finalmente, el uso del operador módulo en la lógica de navegación garantiza secuencias
    cíclicas y estables. Esta metodología resultante proporciona una guía replicable para construir
    aplicaciones fluidas y coherentes en pantallas de gran formato.
\end{abstract}

\begin{IEEEkeywords}
    Android TV, Jetpack Compose, Kotlin, Smart TV.
\end{IEEEkeywords}

% --- ABSTRACT (Inglés) ---

\vspace{0.5cm}
\renewcommand\abstractname{Abstract}
\begin{abstract}
    This research establishes a structured methodology for developing applications with sequential visual
    transitions within the Android TV environment. The work addresses the fragmentation of the Smart TV
    market and the need for development approaches that ensure performance and compatibility. The
    proposed methodology employs Kotlin and the Jetpack Compose for TV framework, which are
    recommended by Google, and is validated through an image carousel prototype. The implementation
    demonstrated the importance of reactive state management (remember, mutableStateOf) in maintaining
    the stability of visual transitions. The effectiveness of asynchronous loading of remote resources (Coil
    library) was confirmed to ensure a continuous visual flow, avoiding main thread blocking. Finally, the use
    of the modulo operator in the navigation logic guarantees cyclic and stable sequences. This resulting
    methodology provides a replicable guide for building fluid and consistent applications on large-format
    screens.
\end{abstract}

\begin{IEEEkeywords}
    Android TV, Jetpack Compose, Kotlin, Smart TV.
\end{IEEEkeywords}

% --- INTRODUCCIÓN ---
\section{Introducción}
Las Smart TV han transformado el entretenimiento doméstico... (PEGA AQUÍ TU TEXTO CORREGIDO)...

\subsection{Planteamiento del Problema}
La disparidad de hardware y la gestión ineficiente... (PEGA EL PLANTEAMIENTO)...

\subsection{Justificación}
Esta investigación es necesaria para estandarizar... (PEGA LA JUSTIFICACIÓN)...

\subsection{Tipo de Investigación}
Se realizó una investigación aplicada con enfoque experimental...

% --- METODOLOGÍA ---
\section{Metodología}
La metodología adoptada para esta investigación se divide en tres fases secuenciales...

\subsection{Fase 1: Entorno Experimental}
Para garantizar la reproducibilidad... (PEGA AQUÍ LO DE MATERIALES)...

\subsection{Fase 2: Configuración del Dispositivo de Prueba}
Se configuró un entorno de virtualización... (PEGA AQUÍ LO DE LA CONFIGURACIÓN DEL EMULADOR)...

\subsection{Fase 3: Implementación de la Arquitectura}
La arquitectura base se fundamenta en... (PEGA AQUÍ LO DEL CÓDIGO BASE)...

% --- RESULTADOS ---
\section{Resultados}

\subsection{Validación de la Propuesta}
Como resultado del proceso de investigación... (PEGA AQUÍ EL INICIO DE RESULTADOS)...

% --- CONCLUSIONES ---
\section{Conclusiones}
En conclusión, esta investigación logró establecer... (PEGA TUS CONCLUSIONES)...


\end{document}
